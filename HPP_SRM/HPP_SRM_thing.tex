%---------------------------------------------------------------------
%
%                          HPP_SRM_thing
%
%---------------------------------------------------------------------


\chapter*{Una base de datos para recoger resultados de proteomica dirigida}
\addcontentsline{toc}{chapter}{Una base de datos para recoger resultados de proteomica dirigida}


El Proyecto Proteoma Humano, \ac{HPP} es un proyecto internacional que 
tiene por objetivo caracterizar los 20.300 genes del genoma humano creando
un mapa de todas las prote�nas codificadas por �stos para 
convertirse as� en un gran recurso global que permita, al igual que ocurri�
con el Proyecto Genoma Humano, desarrollar sobre esta base nuevas investigaciones
aplicadas al avance en diagn�stico y tratamiento de enfermedades.

El HPP se ha estructurado en dos programas.
Por una parte, el programa llamado de Biolog�a y Enfermedad, \ac{BDHPP} 
constituye la parte de investigaci�n aplicada,
coordina las contribuciones por parte de los grupos de investigaci�n implicados 
\citep{Aebersold2013} que estudian prote�nas relacionadas con distintas
enfermedades como diabetes, c�ncer o enfermedades infecciosas.

Por otra parte, el programa Cromosoma-c�ntrico \ac{CHPP} es la parte m�s b�sica y estructural,
persigue definir el conjunto de las prote�nas codificadas en cada uno de los cromosomas.
\citep{Paik2012}. La estrategia que se ha empleado en este programa
ha sido la de dividir esta tarea de 
forma que cada pa�s o consorcio integrante del proyecto \textit{adopte} un cromosoma
para identificar sus prote�nas. 

Ambos programas se sustentan idealmente en tres pilares b�sicos que soportan
y son esenciales para el avance en el proyecto, la espectrometr�a de masas, los anticuerpos
y las bases de datos.

La participaci�n espa�ola en el proyecto HPP se ha materializado en la 
creaci�n de un consorcio encargado de caracterizar las prote�nas del
cromosoma 16
