%---------------------------------------------------------------------
%
%                      resumen.tex
%
%---------------------------------------------------------------------
%
% Contiene el cap�tulo del resumen.
%
% Se crea como un cap�tulo sin numeraci�n.
%
%---------------------------------------------------------------------

\chapter{Resumen}
\cabeceraEspecial{Resumen}

%\begin{FraseCelebre}
%\begin{Frase}

%\end{Frase}
%\begin{Fuente}

%\end{Fuente}
%\end{FraseCelebre}

La Prote�mica a Gran Escala, tambi�n llamada de Alto Rendimiento, basada
en la separaci�n de prote�nas (mediante electroforesis) y de
p�ptidos (mediante cromatograf�a) seguida de espectrometr�a de masas en 
t�ndem, es la t�cnica fundamental de la Prote�mica moderna.
En los experimentos de Prote�mica, a continuaci�n de la parte experimental
que conduce a la adquisici�n de los espectros, los resultados se obtienen
mediante la identificaci�n de los p�ptidos y la inferencia de las prote�nas originarias
presentes en las muestras. Para ello, los distintos motores de b�squeda
emplean diferentes estrategias y a continuaci�n generalmente se eval�an
estad�sticamente los resultados mediante m�todos como la Tasa de Falsos
Descubrimientos, FDR o modelos estad�sticos m�s complejos.

La presencia de resultados de experimentos de Prote�mica en repositorios
p�blicos para el hongo pat�geno oportunista \textit{Candida albicans} 
eran hasta hace poco muy escasos, originados en instrumentos de baja resoluci�n y 
por tanto no muy fiables.
As�, el desarrollo de bases de datos y adopci�n de formatos est�ndar en 
Prote�mica juegan un papel esencial para analizar, comparar y presentar
resultados. Las herramientas inform�ticas descritas en esta tesis contribuyen
a esos objetivos.

La base de datos y herramienta web Proteopathogen

\endinput
% Variable local para emacs, para  que encuentre el fichero maestro de
% compilaci�n y funcionen mejor algunas teclas r�pidas de AucTeX
%%%
%%% Local Variables:
%%% mode: latex
%%% TeX-master: "../Tesis.tex"
%%% End:
