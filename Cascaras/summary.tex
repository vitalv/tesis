%---------------------------------------------------------------------
%
%                      resumen.tex
%
%---------------------------------------------------------------------
%
% Contiene el capítulo del resumen.En Inglés
%
% Se crea como un capítulo sin numeración.
%
%---------------------------------------------------------------------

\chapter{Summary}
\cabeceraEspecial{Summary}

%\begin{FraseCelebre}
%\begin{Frase}

%\end{Frase}
%\begin{Fuente}

%\end{Fuente}
%\end{FraseCelebre}
%\subsection*{Desarrollo de herramientas bioinformáticas para Proteómica de alto rendimiento}

\subsubsection*{Introduction}
The concept of Proteomics, a term coined in analogy to Genomics,
was first used by Marc Wilkins in the mid 90s to describe
the total set of proteins being expressed by the genes of a cell,
tissue or organism.
Earlier, in the late 80s, the development of soft ionizations techniques, 
such as Electrospray Ionization (ESI) and Soft Laser Desorption (SLD)
enabled the ability to ionize large bio-molecules such as proteins while
keeping them relatively intact. This settled the foundation upon which
mass spectrometry was applied to what would later become modern proteomics.

In \textit{shotgun} proteomics, the first step of the experiment usually
consists of a digestion of the sample proteins into peptides by means of
a proteolytic enzyme such as trypsin. This greatly increases performance
in terms of the number of proteins that can be detected in a single 
LC-MS/MS run compared to gel-based approaches, but comes at the cost 
of a great complexity at the peptide level and the protein inference problem. 

Peptides are then separated by liquid chromatography (LC), then ionized
and eventually enter the mass spectrometer where they are separated
as a function of their mass-to-charge (\mz) ratio, recorded in the MS\textsuperscript{1} spectrum.
In tandem mass spectrometry (MS/MS)
peptides with higher intensities are selected to be fragmented so 
MS/MS spectra, a collection of \mz values and intensities of the precursor and product ions, are produced.

Once the empirical spectra are acquired, the computational analysis starts.
The most efficient peptide identification method is based on searching 
the acquired spectra against protein sequence databases. This is what
search engines do. Basically, a score measuring the degree of similarity 
between the empirical spectrum and a theoretically derived spectrum (corresponding to a known 
sequence) is given to pairs of spectrum - peptide sequence named PSMs (Peptide to Spectrum Match)

To assess confidence in peptide identification, statistic measures such
as p-values and e-values are given to PSMs. But in the context of a large
experiment generating thousands of MS/MS spectra, further filtering or additional
statistical procedures may be applied.

The estimation of the False Discovery Rate is a simple yet effective procedure. By comparing the 
spectra to artificially generated sequences, called decoys, a reference populaton 
of incorrect matches is created and is assumed to be equivalent to the populaton
of incorrect matches to real sequences (false positives).

Other post-processing method to evaluate confidence in identifications of MS/MS spectra 
is probability mixture modelling, implemented in tools such as PeptideProphet. It 
first creates a discriminant, search engine-independent score and then generates
the distribution of all best matches to the total amount of MS/MS in the experiment.
This distribution is assumed to be a mixture of the correctly assigned and incorrectly assigned
PSMs. Then, using curve-fitting and the Expectation-Maximization algorithm, the correct
and incorrect distributions can be inferred. And finally, using bayesian statistics
a probability of being correct is computed for each PSM.

Then there is the protein inference problem which has two main aspects.
On the one hand there is the non-random grouping of peptides into proteins, 
which propagates error affecting FDR levels. And on the other hand there is 
peptide degeneracy, that is, conserved sequences in different proteins.
Different software tools deal with these issues in different approaches. 
ProteinProphet, used for a large parte of the results in this thesis, recomputes
PSM probabilities rewarding those corresponding to \emph{sibling} peptides, 
and giving weights depending on sequence uniqueness/degeneracy. 
In this way, a minimal list of proteins that explains all observed peptides
is provided along with the corresponding protein-level probability values.

Fundamental to modern proteomics is also data sharing and dissemination. 
Public on-line repositories, like PRIDE and PeptideAtlas play a key role in 
this sense. Unlike PRIDE, PeptideAtlas, reprocesses all MS/MS spectra through
its specific Trans Proteomic Pipeline to ensure data uniformity and quality.
In addition, the Proteomics Standard Initiative has developed standard data formats 
to promote sharing and re-analysis of experimental data. In particular,
MzIdentML, the standard for peptide and protein identification,
 is used as input data format for Proteopathogen, developed in this thesis.



\subsubsection*{Objectives}
The presence of proteomic data related to the fungal opportunistic 
pathogen \textit{C. albicans} in online public repositories was
until recently very sparse, sometimes originated from low resolution
instruments and therefore seldom reliable.
In this context, the development of databases and the adoption
of standard formats in proteomics have an essential role in the analysis,
sharing and dissemination of results. The software tools presented here 
contribute to these objectives.

\subsubsection*{Results}
The database and web tool Proteopathogen was, in its original version
\citep{Vialas2009b},
the first described software tool
that combined proteomics experiments results with specific information 
relevant to the study of \textit{C.albicans} proteins such as GO (Gene Ontology)
terms and KEGG (Kyoto Encyclopedia of Genes and Genomes) pathways annotations,
all available through a rich web interface.
The Proteomics Standard Initiative promoted format for peptide and protein
identifications MzIdentML had not been released yet, so Proteopathogen
collected results in tab separated text formats dependant on the software
that was used to generate them.

Following the first version, the software was adapted to make use of
mzIdentML, as its source of data
\citep{Vialas2015} and the database was populated with quality data from PeptideAtlas.

Proteopathogen has become a versatile online tool to allow
visualization and analysis of proteomics experiments results by the wet-lab 
users but also enabling revision by reviewers of the journals of the field.



With the development of a \textit{C. albicans} PeptideAtlas, for the
first time a fungal pathogen has been included in the global PeptideAtlas project.
In its original build \citep{Vialas2013}, the \textit{C. albicans} PeptideAtlas provides identification
results for over 2500 proteins at 1.2\%FDR, which represent a coverage of 41\% of
the predicted proteome. Then two 
experiments, consisting of an extensive subcellular fractionation and an off-gel
peptide-level fractionation from different growing conditions, were specifically
designed to increase proteome coverage. These, together with two additional datasets
on the surface, and the vesicles and secreted proteomes were reprocessed along with
the datasets in the previous build to generate a new version of the PeptideAtlas.
In addition, for the first time in a large-scale \textit{C.albicans} project
a database with allele-specific sequences was used. 

This reprocessing, combining results from three different search engines,
effectively means the most exhaustive \textit{C. albicans}
proteome characterization up to the current date.
It describes over 71000 detected peptides assigned to 4174 proteins, which
represent 66\% of the predicted proteome. 

Moreover, through its web interface, this PeptideAtlas enables a valuable
resource to aid in the selection of candidate proteotypic peptides, the
first, essential step in targeted proteomics assays.

Lastly, and in order to favor intercommunication between resources CGD and
PeptideAtlas, we have contacted the CGD developers and promoters providing 
them with links so, through the \textit{protein} tab in CGD, users
may access the corresponding identification data in PeptideAtlas.




\subsubsection*{Conclusions}

The developed database and web application called Proteopathogen has been proven to be a
valuable, greatly useful tool to view and analyze proteomics experiments 
results using \textit{C. albicans} as a model for the study of fungal pathogens.

With the adoption of the mzIdentML standard as input data format for new experiments
in Proteopathogen, a solid foundation is established to ensure continuity and easing
the incoroporation of new results using this format which is independent of the
experimental and computational procedures that may have generated the results

The created \textit{C. albicans} PeptideAtlas has established for the first time
a large-scale characterization of the proteome for a model of fungal pathogen
in the PeptideAtlas global project.


This \textit{C. albicans} PeptideAtlas describes 71310 peptides and 4174 proteins
(for a 0.10\% FDR at PSM level), means the most exhaustive proteome characterization
for this organism (66\%), provides information about the original alleles
and it currently is the most complete and reliable resource available.

Furthermore, the \textit{C. albicans} PeptideAtlas describes 2860 proteins
for which their corresponding ORFs are termed \textit{uncharacterized} because
they lack a known protein product, that is 63\% of them.

\endinput
% Variable local para emacs, para  que encuentre el fichero maestro de
% compilación y funcionen mejor algunas teclas rápidas de AucTeX
%%%
%%% Local Variables:
%%% mode: latex
%%% TeX-master: "../Tesis.tex"
%%% End:
