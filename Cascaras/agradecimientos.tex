%---------------------------------------------------------------------
%
%                      agradecimientos.tex
%
%---------------------------------------------------------------------
%
% agradecimientos.tex
% Copyright 2009 Marco Antonio Gomez-Martin, Pedro Pablo Gomez-Martin
%
% This file belongs to the TeXiS manual, a LaTeX template for writting
% Thesis and other documents. The complete last TeXiS package can
% be obtained from http://gaia.fdi.ucm.es/projects/texis/
%
% Although the TeXiS template itself is distributed under the 
% conditions of the LaTeX Project Public License
% (http://www.latex-project.org/lppl.txt), the manual content
% uses the CC-BY-SA license that stays that you are free:
%
%    - to share & to copy, distribute and transmit the work
%    - to remix and to adapt the work
%
% under the following conditions:
%
%    - Attribution: you must attribute the work in the manner
%      specified by the author or licensor (but not in any way that
%      suggests that they endorse you or your use of the work).
%    - Share Alike: if you alter, transform, or build upon this
%      work, you may distribute the resulting work only under the
%      same, similar or a compatible license.
%
% The complete license is available in
% http://creativecommons.org/licenses/by-sa/3.0/legalcode
%
%---------------------------------------------------------------------
%
% Contiene la p�gina de agradecimientos.
%
% Se crea como un cap�tulo sin numeraci�n.
%
%---------------------------------------------------------------------

\chapter{Agradecimientos}

\cabeceraEspecial{Agradecimientos}

\begin{FraseCelebre}
\begin{Frase}
Frase c�lebre que vaya bien en los agradecimientos
\end{Frase}
%\begin{Fuente}
%\end{Fuente}
\end{FraseCelebre}


Quiero dar las gracias en este espacio a toda la gente que ha contribuido
directa o indirectamente, incluso inconscientemente, a que yo finalmente,
han pasado unos cuantos a�os ya, pueda haber escrito esta tesis.
Estos agradecimientos son una especie da carta abierta para que la lean
todas estas personas, a las que me dirigir� en segunda persona. Cada cual
que se sienta aludido o aludida en su turno. Tambi�n habr� qui�n 
est� aqu� mencionado pero no lea esto. Bueno, ah� queda escrito.

En primer lugar, Ana, gracias por compartir tu vida conmigo, y tu manera de verla,
tan optimista y alegre, sin t� no habr�a podido.

...aqui m�s cosas de ana...

Alberto Pascual, gracias por animarme y empujarme en mis primeros pasos 
en la bioinform�tica en mi primera etapa en el CNB, gracias a t� encontr�
mi sitio en Farmacia donde he hecho todo este trabajo. 

Concha, por supuesto, �que gran jefa!, exigente pero comprensiva, gracias
a t� he podido aprender un mont�n de cosas, muchas de Prote�mica, pero tambi�n de 
la vida en general. A Luc�a y Gloria, gracias tambi�n. A veces creo
que vosotras tambi�n podr�ais haber sido mis directoras. Me hab�is aportado 
muchas ideas de genuinas cient�ficas. Os respeto y admiro.

Mis compis de la Unidad 1. Por aqu� ha desfilado un mont�n de gente. Todos me hab�is ayudado en 
cosas de trabajo, pero tambi�n a hacer todo el tiempo que he pasado aqu� (que no es poco) mucho 
m�s agradable. Pero de todos hay algunos m�s fijos de la U1. Jose, el t�o m�s eficiente en el 
trabajo que conozco, �qui�n no querr�a contratarte?, espero
que pronto encuentres un buen post-doc.
Aida, 
Seguir� con otros cl�sicos.
Claudia,
Carolina,
Vir
Y por �ltimo los nuevos compis de la U1, Perce, Catarina, Ahinara


\endinput
% Variable local para emacs, para  que encuentre el fichero maestro de
% compilaci�n y funcionen mejor algunas teclas r�pidas de AucTeX
%%%
%%% Local Variables:
%%% mode: latex
%%% TeX-master: "../Tesis.tex"
%%% End:
