%---------------------------------------------------------------------
%
%                         Conclusiones
%---------------------------------------------------------------------

\chapter*{Conclusiones}
%mirar problema con fancyheader y secciones no numeradas, pagina 21-22 del manual
%\fancyhead[RO,LE]{\sc{Conclusiones}}
\addcontentsline{toc}{chapter}{Conclusiones}



\begin{enumerate}

\item La base de datos y aplicaci�n web Proteopathogen
ha demostrado ser una herramienta de gran utilidad para la visualizaci�n
y an�lisis de resultados de prote�mica en experimentos que usan 
\textit{Candida albicans} como organismo modelo de estudio de hongos pat�genos.

\item La adopci�n del est�ndar mzIdentML como formato de origen para
incorporar nuevos datos en Proteopathogen asegura la estabilidad y futuro
de este proyecto facilitando la incorporaci�n de resultados procedentes 
de nuevos experimentos.

\item El PeptideAtlas de \textit{Candida albicans} desarrollado
supone la caracterizaci�n m�s exahustiva del proteoma de este organismo.
y es el recurso m�s completo y fiable disponible p�blicamente.



\end{enumerate}
