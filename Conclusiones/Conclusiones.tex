%---------------------------------------------------------------------
%
%                         Conclusiones
%---------------------------------------------------------------------

\chapter*{Conclusiones}
%mirar problema con fancyheader y secciones no numeradas, pagina 21-22 del manual
%\fancyhead[RO,LE]{\sc{Conclusiones}}
\addcontentsline{toc}{chapter}{Conclusiones}



\begin{enumerate}

\item La base de datos y aplicaci�n web creada, denominada Proteopathogen,
es una herramienta p�blica \textit{online} de gran utilidad para la visualizaci�n
y an�lisis de resultados de prote�mica en estudios que usan 
\textit{C. albicans} como organismo modelo de hongos pat�genos.

\item La adopci�n del est�ndar mzIdentML como formato de origen para
incorporar nuevos datos en Proteopathogen asegura la estabilidad y futuro
de este proyecto ya que es posible obtener archivos con resultados
de identificaciones en este formato independientemente del procesamiento experimental
y computacional.

\item Se ha creado un PeptideAtlas de \textit{C. albicans} estableciendo
por primera vez una caracterizaci�n a gran escala del proteoma
de un modelo de hongo pat�geno en el proyecto global PeptideAtlas.


\item El PeptideAtlas de \textit{C. albicans} describe 71310 p�ptidos y 4174
prote�nas (para un FDR de 0,10\% a nivel de PSM),
supone la caracterizaci�n m�s exahustiva del proteoma de este organismo (66\%)
y es el recurso m�s completo y fiable disponible p�blicamente.

\item En el PeptideAtlas de \textit{C. albicans} se describen 2860 prote�nas
para las que sus correspondientes ORFs se denominan
\textit{uncharacterized} por carecer de un producto g�nico conocido, lo que supone un 63\% de �stos.



\end{enumerate}
