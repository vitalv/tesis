%---------------------------------------------------------------------
%
%                         Discusion
%---------------------------------------------------------------------

\chapter*{Discusi�n}
%mirar problema con fancyheader y secciones no numeradas, pagina 21-22 del manual
\fancyhead[RO,LE]{\sc{Discusi�n}}

\addcontentsline{toc}{chapter}{Discusi�n}
%\cabeceraEspecial{Discusion}


\phantomsection 
\section*{Desarrollo de una aplicaci�n web para datos de prote�mica a gran escala de \textit{Candida albicans}}
\addcontentsline{toc}{section}{Desarrollo de una aplicaci�n web para datos de prote�mica a gran escala de Candida albicans}
%-------------------------------------------------------------------
\label{cap1:sec:Desarrollo de una aplicaci�n web para datos de prote�mica a gran escala de Candida albicans}


Primero, contexto, background

Addressing proteomic studies involving the way Candida
interacts with immune cells is thus essential in order to
improve our comprehension of the process of infection and
represents the primary step of investigation that could lead
to future development of diagnosis methods, vaccines and
antifungal drugs


Segundo. Sin embargo, a pesar de la importancia clinica y de los
estudios prote�micos

Los repositorios p�blicos de resultados de identificaciones de
de prote�nas en experimentos relacionados con Candida eran muy escasos
y diseminados.

Tercero. En ese contexto, para resolver este tipo de escasez/ausencia
se desarrollaron Proteopathogen y el PeptideAtlas


Proteopathogen represents, up to date, the
first public web-based repository for proteomics data related
to studies involving C. albicans pathogenicity and its inter-
action with immune system cells in the host. Moreover, it
enables a framework for public access and submission of
this type of data and it is intended to be more actively
populated in the near future, including data from different
pathogenic fungi and mammalian cells, becoming a refer-
ence database in its field. Unlike other protein identification
databases, Proteopathogen is focused to a specific topic but,
at the same time, includes a wide range of data including
descriptions of the experimental contexts, information on
proteins such as GO and pathway annotations, structural
information and detailed MS parameters. Therefore,
Proteopathogen will contribute to save time and facilitate
