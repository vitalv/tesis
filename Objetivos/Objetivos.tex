%---------------------------------------------------------------------
%
%                         Objetivos
%---------------------------------------------------------------------

\chapter*{Objetivos}
%mirar problema con fancyheader y secciones no numeradas, pagina 21-22 del manual
%\fancyhead[RO,LE]{\sc{Objetivos}}

\addcontentsline{toc}{chapter}{Objetivos}
%\cabeceraEspecial{Objetivos}


\begin{itemize}
  \item Desarrollo de una aplicaci�n web, denominada Proteopathogen,
  apoyada en una base de datos
  para recoger, almacenar y visualizar resultados de identificaci�n de 
  p�ptidos y prote�nas procedentes de estudios prote�micos
  relacionados con la interacci�n hospedador-pat�geno usando
  \mbox{\textit{C. albicans}} como modelo de hongo pat�geno.
  \item Adopci�n del formato est�ndar de identificaciones de p�ptidos y prote�nas mzIdentML
  como fuente �nica de informaci�n para la base de datos y aplicaci�n web \mbox{Proteopathogen} para
  que la inserci�n de los datos sea independiente del procesamiento experimental y computacional empleado.
  \item Creaci�n de un atlas pept�dico o PeptideAtlas para \textit{C. albicans}
  \begin{itemize}
    \item Recopilaci�n de resultados de espectrometr�a de masas   %no resultados de identificaciones, sino los espectros
    y creaci�n de una primera versi�n de \mbox{PeptideAtlas} empleando el flujo
    de trabajo proporcionado por las herramientas que conforman TPP (\textit{Trans Proteomic Pipeline}) que 
    incluye conversi�n de los espectros a un formato est�ndar, identificaci�n de p�ptidos, inferencia
    de prote�nas y validaci�n estad�stica de los resultados. 
    \item Desarrollo de nuevos experimentos dise�ados \textit{ad hoc} para incrementar la 
    cobertura del proteoma. Implementaci�n de nuevas rutinas de an�lisis incorporando
    al flujo de trabajo TPP una nueva base datos con secuencias espec�ficas de alelo y un 
    procesamiento multi-algoritmo.
  \end{itemize}
\end{itemize}
