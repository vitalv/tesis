%---------------------------------------------------------------------
%
%                         Objetivos
%---------------------------------------------------------------------

\chapter*{Objetivos}
%mirar problema con fancyheader y secciones no numeradas, pagina 21-22 del manual
\fancyhead[RO,LE]{\sc{Objetivos}}

\addcontentsline{toc}{chapter}{Objetivos}
%\cabeceraEspecial{Objetivos}


\begin{itemize}
  \item Desarrollo de una aplicaci�n web, denominada Proteopathogen (v 1.0)
  respaldada por una base de datos
  para recoger, almacenar y analizar resultados de identificaci�n de 
  prote�nas procedentes de experimentos de prote�mica
  relacionados con \textit{Candida albicans}
  \item Adopci�n del formato est�ndar de identificaciones mzIdentMl 
  como fuente de datos para la base de datos y aplicaci�n web Proteopathogen (v 2.0)
  \item Recopilaci�n de resultados de espectrometr�a de masas para la  %no resultados de identificaciones, sino los espectros
  creaci�n de un PeptideAtlas, o Atlas Peptidico, de \textit{Candida albicans}
  \item Rean�lisis de los datos existentes en el PeptideAtlas original
  usando una nueva base de datos con secuencias espec�ficas de alelo, un
  un procesamiento multi-algoritmo y resultados de nuevos experimentos dise�ados \textit{ad hoc}
  para mejorar la cobertura del proteoma.
  %\item Creacion de una base de datos para recoger y almacenar metodos
  %de proteomica dirigida (MRM) empleados para detectar proteinas en el 
  %contexto del consorcio espaol dedicado al mapeo cromosoma-centrico en el proyecto proteoma humano (HPP) 
\end{itemize}
