%---------------------------------------------------------------------
%
%                         Introduccion
%---------------------------------------------------------------------


\chapter{}
%Quiero la estructura de un capitulo pero que no ponga "Capitulo I"

\cabeceraEspecial{Introduccion}

%\begin{FraseCelebre}
%\begin{Frase}
%...
%\end{Frase}
%\begin{Fuente}
%...
%\end{Fuente}
%\end{FraseCelebre}

%\begin{resumen}
%...
%\end{resumen}


%-------------------------------------------------------------------
\section{Proteomica. Conceptos Basicos}

%-------------------------------------------------------------------
%\label{cap1:sec:introduccion}


%-------------------------------------------------------------------
\section{Espectrometria de masas. Conceptos Basicos}



%-------------------------------------------------------------------
\section{Proteomica en gel}



%-------------------------------------------------------------------
\section{Proteomica shotgun}



%-------------------------------------------------------------------
\section{Asignacion Peptido-Espectro}



%-------------------------------------------------------------------
\section{Proteomica dirigida}





%-------------------------------------------------------------------
%\section*{\NotasBibliograficas}
%-------------------------------------------------------------------
%\TocNotasBibliograficas

%Citamos algo para que aparezca en la bibliografía\ldots
%\citep{ldesc2e}

%\medskip

%Y también ponemos el acr

%-------------------------------------------------------------------
%\section*{\ProximoCapitulo}
%-------------------------------------------------------------------
%\TocProximoCapitulo

...

% Variable local para emacs, para  que encuentre el fichero maestro de
% compilación y funcionen mejor algunas teclas rápidas de AucTeX
%%%
%%% Local Variables:
%%% mode: latex
%%% TeX-master: "../Tesis.tex"
%%% End:
